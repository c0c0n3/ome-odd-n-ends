\section{Software Architecture}

Each of the coming subsections provides an architectural view\cite{soni:views,
kruchten:views}, addressing a specific facet of the Background Import; taken 
together, these views convey just enough information to describe the architecture
of the proposed solution at a very-high level.\footnote{A more detailed software 
architecture document may be provided during the course of the project if the need
for it arises.}
The reader is assumed to be familiar with the OMERO platform and its implementation
at MRI---deployment, operation, network topology, etc.

\subsection{Requirements}
As already discussed in the introduction, the key requirement is to untie the image
import from the login session. Whatever the mechanism to achieve this, the transfer
quality, in terms of reliability and performance, should not degrade.
Resource usage (CPU, I/O, etc.) on the acquisition workstation should not be any
higher than that of the OMERO importer and microscope operation should not be affected.
Additionally, robust failure handling and recovering has to be in place so that users
and system administrators are notified of failures and failed imports can be retried 
or resumed as appropriate; successful imports have to be notified to the users who 
requested them. Data integrity must be preserved and it should not be possible to 
remove a file while it is being imported. 

\subsection{Functional View}
The idea is to have a new background process, the Import Proxy, that can run an image 
import on behalf of OMERO clients; this process would run outside of any user session
so to untie the image data transfer from login sessions. 

In detail, the Import Proxy is a Java ICE servant exposing an asynchronous call that 
accepts the exact same inputs as those the OMERO client currently passes to the OMERO
server to perform an import (object hierarchy, annotations, etc.) except for the image
data which are instead replaced by a pointer to the actual image file. 
The Import Proxy runs the import in the exact same way as the OMERO Java client does 
presently---i.e. save inputs to OMERO server, parse and transfer image file. 
The proxy process needs to have access to the files to import which are therefore 
expected to be available either from the local file system, if the client and the 
proxy run on the same machine, or via a network share, if the proxy is on a different
machine.
Files will be locked for the entire duration of the import in order to prevent users
from accidentally deleting them before import completion; further data integrity is 
already enforced by the current OMERO import functionality---e.g. SHA checksums. 
Upon successful completion, the Import Proxy sends an email notification to the user
who requested the import; if the import fails, both the user and system administrator
are notified. A write-ahead log provides support for recovering from failures so
that broken imports may be fixed, e.g. by retrying them if at all possible or by
requesting the system administrator's intervention.

Note that because the Import Proxy runs an import in the exact same way as the 
current OMERO client does, transfer quality and resource usage should be the same 
as what they presently are (i.e. no degradation) if the proxy and the client both
run on the same host.

In order to be able to use the Import Proxy, the OMERO Java client is extended with
a new Import Plugin. This plugin is configurable so that Background Import functionality 
will not be available for sites that do not require it---this is the default 
configuration setting. The plugin provides a new button that users, after going 
through the usual import workflow, can press to trigger a background import in which
case the plugin calls the Importer Proxy and returns immediately; the user can then 
quit the OMERO client and log out while the import is run in the background.


\subsection{Development}
The Background Import will be part of the OMERO Java client (Insight) project. All 
the Background Import code will be developed in the OMERO Java client code base. Two 
separate modules will be added: one for the Import Proxy and one for the Import Plugin.
Build and configuration management are already in place for the entire OMERO Java client
code base, so the current infrastructure will be adopted wholesale and existing coding 
standards will be adhered to. 
Automated testing is available too, so unit and integration tests will be developed and
added to the current automated test suite.  
The Background Import byte-code is also packaged inside the existing OMERO Java client 
install bundle.

Note that sharing of project infrastructure and deployment is advantageous in terms
of speed of delivery but is not necessarily the best arrangement in terms of release
management and deployment dependencies. So this may change in the future and the
Background Import could become a separate OMERO project.


\subsection{Deployment}
In our deployment configuration, the OMERO Java client is installed on each acquisition
workstation. The client is configured to use the Import Proxy which runs as a service
on the same workstation. The service is set up using a control script included in the
client install bundle; the script starts (stops/restarts) a Java process to run the
Import Proxy code which is also part of the same client bundle. 
Thus, only one installation is needed to run both the client and the proxy. However,
whereas the client is available to all workstation users, the proxy is a single 
background process running on the workstation to service all users. The proxy uses
whichever OMERO server the client is configured to use. 
Run-time, hardware, and network requirements for our deployment configuration are 
the same as those of a standard OMERO Java client deployment.


\subsection{Operation}
The installation procedure entails installing an OMERO Java client on each acquisition
workstation (as currently done), configuring it to use the Import Proxy, and setting
up the Import Proxy to run as a service using the provided control script. 
Additionally, the account under which the proxy service runs needs to have enough 
permissions to read any image produced by the microscope attached to the workstation.
Upgrade procedures for the Background Import components are exactly the same as for
an OMERO Java client deployment.

The proxy service should be monitored (e.g. added to the list of processes monitored
by system administrators) to ensure its availability. If the service is down, any user
(of that workstation) trying to import an image would receive an error and would not 
be able to import.
Even though this can be remedied in minutes just by manually restarting the service, 
it is best to try and prevent errors by automated monitoring and recovery in order to
raise the quality of service.
Any failure encountered by the Import Proxy while running an import will be emailed
to a configured administrator mailbox which system administrators should make sure
to have access to.

User training and documentation will be provided. For support and assistance, users
can contact system administrators.
