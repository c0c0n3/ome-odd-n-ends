
\section{Introduction}

\subsection{Background and Context}
Montpellier RIO Imaging (MRI) is a multi-site imaging core facility with over 700 active
users and more than 60 acquisition systems.
Due to the sheer size and complexity of the structure, the staff is facing an increasing
number of data management and analysis challenges. To overcome these challenges, MRI
decided to adopt OMERO\cite{ome:omero-pub,ome:omero-www} as the main infrastructure for
the storage, management, and analysis of image data. There are also plans to participate
in the development of OMERO to better serve the specific needs of the local imaging 
community.

OMERO software has been deployed to most acquisition workstations within MRI so that
after images have been produced by a microscope, experimenters can import them directly
from an acquisition workstation into the OMERO image repository. Of course, they still
have an option to copy the image data to some other storage facility of their own choice. 
Either way, data will have to be transferred out of the workstation as acquired images 
are periodically deleted from local storage to make room for new ones. Whichever way
experimenters choose to transfer the data currently requires them to be logged on the 
acquisition workstation until the image data have been fully transferred out of the
workstation into the designated storage location. 

This arrangement has the following disadvantages:
\begin{itemize}
\item \emph{Lessened microscope availability}. A microscope is not available to other
 users until the experimenter has logged out of the acquisition workstation so the next
 user will have to wait until the transfer is finished. 
\item \emph{Additional costs}. Because experimenters are billed for the amount of time 
 they are logged on an acquisition workstation, the time it takes to transfer the data 
 is billed too. 
\end{itemize}

Note that these issues are not inherent to the OMERO import workflow or, more generally,
to the data transfer mechanism per se, but rather are caused by the fact that the  
\emph{data transfer is tied to a user session}.


\subsection{Proposed Solution}
This project aims to remove the above limitations. The goals are cost and resource usage
optimisation within microscope facilities as well as demonstrating that this can easily
be accomplished using OMERO technology.

To this end, we propose an extension to the Open Microscopy Environment software 
platform\cite{ome:www}. The idea is to develop a set of software components and 
integrate them into the existing OMERO platform so to allow experimenters to log 
out just after triggering an image import into the OMERO repository. The import 
will then run autonomously to completion, thus making the microscope immediately
available to the next user and avoiding billing for the data transfer.


\subsection{Document Outline}
The reminder of this document provides information about key project areas at a level
of detail suitable to support initial project execution. More in-depth and refined 
documentation will be provided during the course of the project as the need arises.

The next section is devoted to the software architecture. In it, requirements are
stated and key solution aspects, from design to testing to deployment, are described
by means of interlocking views and their relation to requirements.
This information forms the basis for scoping and planning which are discussed in the
following section, the Project Plan. 
Then, in the Methodology section, we consider software process issues, development
activities and tools.
The final sections explore possible avenues for future work and collect some initial
considerations about the project.
